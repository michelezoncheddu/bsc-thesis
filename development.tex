\section{Development}

\subsection{Cache}
\dots

\subsection{Motion capture}
For a better training of the neural network, the main problem was to achieve the cleanest data from the sensors. Previously, the data collected were:
\begin{itemize}
	\item yaw, pitch and roll angles;
	\item accelerations in spherical coordinates system;
	\item raw gyroscope data;
	\item a velocity approximation.
\end{itemize}
\bigbreak

Through preliminary tests has been discovered that acceleration values included gravity; thus it was necessary a more complex training data collection, due to the lack of orientation invariance (in order to obtain such invariance it would have been necessary to collect data with the toy rotated in any possible angle).

Nevertheless, gravity-free acceleration data were computed by the device after the Madgwick fusion algorithm and sent to the server, but only used for the 3D visual representation, both in training and session recording phase. It was decided to keep such data and analyze it to see if it was reliabile over time.
\bigbreak

After some testing, it was discovered that there was an error in the yaw angle calculation that grew with the device rotation over each axis, and it was initially imputed to the Madgwick algorithm, more precisely to the convergence of its error-correction gradient descent algorithm.

Unfortunately, several trials with different algorithm implementations and different values of the $\beta$ parameter (the magnitude of the gyroscope measurement error [CITE]\footnote{Increasing beta leads to faster bias corrections and higher sensitiveness to lateral accelerations.}) have no given better results over time, and worse, sometimes ghost accelerations were found.

During the tests, it was noticed that recordings of the same pattern with different device orientations didn't split the acceleration differently among axes, so the reference system was integral with the Earth's center instead of the device. To align it back to the device, the acceleration vector has been rotated using a 3D rotation matrix.
\bigbreak

A \textit{yaw} is a CCW rotation of $\alpha$ on the $z$-axis. The rotation matrix is
\[
	R_z(\alpha) =
	\begin{pmatrix}
		\cos\alpha & -\sin\alpha & 0 \\
		\sin\alpha & \cos\alpha & 0 \\
		0 & 0 & 1
	\end{pmatrix}
\]
Note that the upper left values compose a 2D rotation matrix, and the coordinates on the third dimension (around whom the rotation happens) are left unchanged. The same applies to the other two matrices, but with the second and the first dimension.
\bigbreak

A \textit{pitch} is a CCW rotation of $\beta$ on the $y$-axis. The rotation matrix is
\[
	R_y(\beta) =
	\begin{pmatrix}
		\cos\beta & 0 & \sin\beta \\
		0 & 1 & 0 \\
		-sin\beta & 0 & \cos\beta
	\end{pmatrix}
\]

A \textit{roll} is a CCW rotation of $\gamma$ on the $x$-axis. The rotation matrix is
\[
	R_x(\gamma) =
	\begin{pmatrix}
		1 & 0 & 0 \\
		0 & \cos\gamma & -\sin\gamma \\
		0 & sin\gamma & \cos\gamma
	\end{pmatrix}
\]
\begin{gather*}
	R(\alpha, \beta, \gamma) = R_z(\alpha) R_y(\beta) R_x(\gamma) = \\
	\begin{pmatrix}
		\cos\alpha \cos\beta & \cos\alpha \sin\beta \sin\gamma - \sin\alpha \cos\gamma & \cos\alpha \sin\beta \cos\gamma + \sin\alpha \sin\gamma \\
		\sin\alpha \cos\beta & \sin\alpha \sin\beta \sin\gamma + \cos\alpha \cos\gamma & \sin\alpha \sin\beta \cos\gamma - \cos\alpha \sin\gamma \\
		-\sin\beta & \cos\beta \sin\gamma & \cos\beta \cos\gamma
	\end{pmatrix}
\end{gather*}
It is important to note that $R(\alpha, \beta, \gamma)$ performs the roll first, then the pitch, and finally the yaw. If the order of these operations is changed, a different rotation matrix would result [CITE].
\bigbreak

Given an acceleration vector $\vec a$, the rotated one is $\vec a_r = R(\alpha, \beta, \gamma) \vec a$.
The multiplication doesn't need specific optimization, because the C++ code is compiled by \texttt{GCC} with the \texttt{-Os}, flag, that performs the loop unrolling when needed; however, the dimensions at stake are very small.
\bigbreak

The velocity approximation was excluded by the training set because\dots
\bigbreak

\subsection{Delay in real-time data plotting}
\dots
