\chapter{Introduction}
MoVEAS (acronym of Monitoring and Visualization of Early Autism Signs) is a project that aims to detect, in a non-invasive way, early signs of autistic spectrum disorders (ASD), by monitoring play activities in young children \cite{Bon20, Lan19}.
\bigbreak

The way children interact with the world is called \textit{play} from adults. Observing children while playing is a widespread technique for diagnosing ASD, especially in early age \cite{Ozo08}, and videotaped play session of groups of children with typical development, against groups of children with ASD, have been proven to be useful for the diagnosis \cite{Bar05, Ozo08, Wet10}.

Children with ASD play in a different way, compared to their neurotypical peers: they use the toys in atypical and restricted ways (e.g. spinning), play with repetitive behaviors, and visually explore objects in unusual ways \cite{Bru07, Ozo08}.
\bigbreak

Diagnosing ASD can be difficult, since there's no medical test to diagnose the disorders, and the \textit{spectrum width} makes it even harder. This project's aim is to simplify the specialists' diagnosis process, by identifying ASD-related behaviors and allowing to indirectly monitor the play session.
\bigbreak

This document will show how the development of the previous project has moved on, how the issues have been detected and solved, which features have been improved and which new ones have been added.

Improvement of the neural network movements classification.
\bigbreak

Divisione della tesi.


Bru07
Restricted object use in young children with autism Definition and construct validity

Wet10
Early Indicators of Autism Spectrum Disorders in the Second Year of Life (affiancare video recordings con highlights)

Autism as a developmental disorder in intentional movement and affective engagement

Bar05
Object Play in Infants With Autism: Methodological Issues in Retrospective Video Analysis

Ozo08
Atypical object exploration at 12 months of age is associated with autism in a prospective sample
