\chapter{Introduction}
MoVEAS (acronym of Monitoring and Visualization of Early Autism Signs) is a project that aims to detect, in a non-invasive way, early signs of autistic spectrum disorders (ASD), by monitoring play activities in young children through sensorized toys and classifying the activities through a neural network \cite{Bon20, Lan19}.
\bigbreak

The way children interact with the world is called \textit{play} from adults. Observing children while playing is a widespread technique for diagnosing ASD, especially in early age \cite{Ozo08}, and videotaped play session of groups of children with typical development, against groups of children with ASD, have been proven to be useful for the diagnosis \cite{Bar05, Ozo08, Wet10}.

Children with ASD play in a differently from to their neurotypical peers: they use the toys in atypical and restricted ways (e.g. spinning), play with repetitive behaviors, and visually explore objects in unusual ways \cite{Bru07, Ozo08}.

Diagnosing ASD can be difficult, since there's no medical test to diagnose the disorders, and the \textit{spectrum width} makes it even harder. This project's aim is to simplify the specialists' diagnosis process, by identifying ASD-related behaviors and allowing to indirectly monitor the play session.
\bigbreak

This document will show how the development of the previous project has moved on, how the issues have been detected and solved, which features have been improved and which new ones have been added.

It will be illustrated the selection of the relevant data to classify the movements, the analysis of the consistency of the data obtained by the data fusion algorithms over time, then the comparison between two new neural network models, and the performance of the final model in a real scenario simulation.
\bigbreak

\paragraph{Thesis outline}
The next chapter describes the background for a better understanding of the rest of the thesis, such as theoretical concepts and practical details. Afterwards, it will be shown the previous version of the project' architecture, to introduce the actual development. Then, the achieved results are illustrated, and the final chapter summarizes the contributions of this work, putting forward suggestions for future developments.
